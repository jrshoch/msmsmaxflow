\documentclass[12pt]{article}

%-------Packages---------
\usepackage{enumerate}
\usepackage[margin=0.75in]{geometry}
\usepackage{rotating}
\usepackage{hyperref}
\usepackage{verbatim}
\usepackage{amsthm}

\bibliographystyle{plain}

\newtheorem{lemma}{Lemma}


%--------Meta Data: Fill in your info------
\title{Max Flow in a Directed Planar Graph}
\date{\today}
\author{Jason Hoch and John Wang \\
6.854 Final Paper}


\begin{document}

\maketitle

\abstract{We implement an max flow algorithm presented by Erikson \cite{erikson2010}.}

\tableofcontents

\newpage

\section{Introduction}

The general maximum flow problem has many applications in operations research and supply chain management. The classic algorithms used to solve the problem are polynomial, but relatively slow. The Edmonds-Karp algorithm requires $O(n m^2)$ time, while Ford Fulkerson requires $O(m \max |f|)$. Improvements made throughout the last few decades have decreased these runtimes by using concepts from blocking-flow and push-relabel algorithms. Dinitz's blocking flow algorithm improved the runtime to $O(n^2 m)$ by using a breadth-first-search in the residual graph to build a layered graph \cite{dinitz1970}. Goldberg and Tarjan \cite{goldbergtarjan1986} introduced a general framework for solving max-flow problems by using the idea of push relabel. Many of the fastest algorithms for maximum flow are variants of this general framework. Recently, Orlin presented an algorithm which runs in $O(nm)$ time. 

A practical application of a maximum flow algorithm, however, would find many of these algorithms prohibitively slow. Even the best known algorithm of Orlin for general graphs is $o(n^2)$ for reasonably connected graphs. However, these algorithms perform poorly due to their generality. In practice, only a certain subset of graphs are likely to arise. For instance, in many applications of maximum flow, it is reasonable to assume that the graphs will be planar. 

This is the case for maximum flow networks on any type of two dimensional map. In these cases, solving the problem in its full generality does too much work, and enforcing constraints on the assumptions of the graph can improve runtime. This paper focuses on the algorithm proposed by Erikson 2010 \cite{erikson2010}, which exploits planarity to achieve $O(n \log n)$ runtime. This significantly reduces the execution time of the algorithm, at least theoretically.

This paper implements the Erikson algorithm and presents the observed runtime on a set of planar graphs. The set of planar graphs generated contains graphs of varying sizes. Our implementation of the Erikson algorithm is tested for how it scales with $n$, the number of vertices in a graph. We attempt to determine an empirical function for how the observed runtime scales with $n$, and compare it with standard library implementations of max-flow algorithms.

The rest of the paper presents a brief overview of the Erikson algorithm, the unique features in our implementation, and the object oriented infrastructure we created. We present the results from our implementation on a set of planar graphs and compare it with standard maximum flow algorithms.

\section{Motivation}

In this section, we provide motivation for our implementation. In particular, we motivate the use of an algorithm specific to planar graphs. We argue that most applications of maximum flow are constrained to planar graphs, and hence, that the Erikson algorithm provides a non-trivial improvement upon past algorithms. 

\subsection{Supply Chain Example}

First, suppose a chain of department stores has a supply chain and is attempting to find the best way to move goods from its production facilities to its stores. The store must transport the goods across the network to its stores, and can stop at any intermediate warehouse in its network. There are a discrete number of warehouses that its trucks can stop at, but there are only a discrete number of trucks that it sends over any route. The goal is to achieve the highest throughput of goods possible from production facilities to its stores.

This problem can be reduced to a single source, single sink maximum flow problem by created a supernode $s$ representing all the production facilities and a supernode $t$ representing the stores. Each production facility will be connected to $s$ and each store will be connected to $t$ with an infinite capacity edge. Finding the maximum flow between $s$ and $t$ will then find the total throughput for a given timespan. Notice that since this transportation problem is defined on land, it is confined to a planar network (assuming that warehouses or rest-stops are located at each path intersection). 

Notice that this type of transportation problem occurs with incredible frequency. Every large chain of stores with a need to move goods across land has a similar problem. Retailers such as Amazon, Walmart, and Target have a large incentive to optimize their transportation networks, as do large shipping companies such as FedEx and UPS. 

\subsection{Computer Vision Example}

There are also many examples of planar max-flow networks which are relevant in computer vision. In particular, finding graph cuts in an image is a widely used industrial application of max-flow. Greig, Porteous, and Seheult \cite{greigporteousseheult1989} provide an example of using max-flow for smoothing noisy images. They show that the maximum a posteriori estimate of a noisy image is given exactly by finding a max-flow in a specifically defined image network. 

The authors in \cite{greigporteousseheult1989} create a network of pixels $i$, each with value $x_i \in \{0,1\}$ corresponding to white and black. Each pixel $i$ in the flow network has neighbors which are the pixels touching it in the actual image. The capacity of the edge between pixels $i$ and $j$ is defined as a log-likelihood ratio corresponding to the probabilitiy that the values of pixels $i$ and $j$ are correct. The paper shows that finding a maximum flow through the network results in a maximum a posteriori estimate of the image. 

The network is clearly planar because the graph can be embedded onto a two dimensional plane (the image). Moreover, \cite{greigporteousseheult1989} spawned a large literature in the computer vision field using max-flows as a means to separate out nosie from images. Other max-flow algorithms in the computer vision field follow in a similar vein to \cite{greigporteousseheult1989}. 

\section{Overview of Erikson's Algorithm}

This section provides an overview of the Erikson algorithm. We summarize the main results from the paper in \cite{erikson2010} and sketch the proofs of the main theorems. The Erikson algorithm relies on the observation that max-flow in directed planar graphs can be reformulated as a parametric shortest paths problem. The algorithm finds a shortest paths spanning tree which whose parameter is perturbed in multiple iterations of the algorithm. Eventually, the resulting tree will arrive at a state where the max-flow can be computed with simple arithmetic. 

\subsection{Definitions}

First, we shall present the fundamental definitions and terminology used throughout the this paper and the Erikson \cite{erikson2010} paper. Note that the notation is the same as in \cite{erikson2010} to reduce confusion. A graph $G = (V,E)$ will be a directed planar graph where each edge $e$ denoted as $u \to v$ contains a head and tail defined as $head(e) = v$ and $tail(e) = u$ respectively. The reversal of the graph is defined as $rev(e) = head(e) \to tail(e)$. 

Each primal graph $G$ has a corresponding dual $G^*$ where each primal face is converted into a dual vertex. Two dual vertices are connected by a dual edge if and only if there is a corresponding primal edge connecting the two faces represented by the two dual vertices. In particular, consider faces $a$ and $b$ in the primal graph, and consider some edge $e$ which separates these two faces. Then the dual graph will contain an edge $e^*$ (the dual of the edge $e$) which is oriented $90^\circ$ counterclockwise to $e$ and connects $a^*$ and $b^*$ in the dual. 

A flow between source $s$ and sink $t$ is defined as a function $\phi : E \to \mathrm{R}$. The flow satisfies antisymmetry and conservation. In particular, the antisymmetric constraint specifies that $\phi(e) = -rev(\phi(e))$ while the conservation constraint specifies that $\sum_{e_w} \phi(e) = 0$ for all edges $e$ such that $w = head(e)$. We define a feasible flow by defining a non-negative capacity function $c : E \to \mathrm{R}$ on each edge $e$. For a flow to be feasible on a graph, we must have $\phi(e) \leq c(e)$ for all edges $e$ in the graph. 

\subsection{Parametric Shortest Paths}

Before moving on to Erikson's algorithm, it is instructive to examine the parametric shortest paths problem, since Erikson's algorithm is fundamentally based on the idea of using parametric shortest paths to compute the max-flow. For this problem, consider a graph $G = (V,E)$ and an additional parameter $\lambda$. We define a new cost function $c : E \to \mathrm{R}$ and obtain a subset $E' \subset E$ of the edges of the graph. The cost function will be set to $c(\lambda, e) = c(e) - \lambda$ for all edges $e \in E'$ and $c(\lambda, e) = c(e)$ for all edges $e \notin E'$. The parametric shortest paths problem asks to compute the largest value of $\lambda$ for which the resulting graph with weights $c(\lambda, e)$ has no negative-weight cycles. 

A number of algorithms that solve this problem use the concept of a pivot tree. Young, Tarjan, and Orlin \cite{youngtarjanorlin1991} solve the problem in $O(nm + n^2 \log n)$ by constructing a single-source shortest paths tree. The distance used in the shortest paths tree is equal to the cost $c(\lambda, e)$ for each edge $e$, given a parameter $\lambda$. The Young, Tarjan, and Orlin algorithm starts increasing $\lambda$ from $-\infty$. At some point, $d(\lambda, v) = d(\lambda, u) + c(\lambda, v)$ for some edge $u \to v$ which is not in the shortest paths tree. When this occcurs, the shortest paths tree incorporates the $u \to v$ edge and removes an appropriate edge $e$ where $head(e) = v$ so that the tree remains a valid shortest paths tree. The algorithm continues until a cycle appears in the tree. Notice that at this point, increasing $\lambda$ by any amount will cause a negative-weight cycle in the original graph because the cycle must be of weight 0. 

This algorithm therefore computes the maximum $\lambda$ for which the graph $G$ has no negative-weight cycles. Erikson's algorithm will use the same concept and apply it to max-flow. In particular, Erikson uses the parameter $\lambda$ to pivot edges in a shortest paths tree. Using a well defined set of weights, one can show that the resulting algorithm solves the max-flow problem.

We also define the dual residual network $G^*_\lambda$ of some graph $G$ for some parameter $\lambda$. It is defined on the dual graph $G^*$ and has a cost function $c : \mathrm{R} \times E^* \to \mathrm{R}$ where $c(\lambda, e^*) = c(\lambda, e)$. Note that the edge $e$ is well defined since each edge $e$ has a corresponding dual edge $e^*$ which is $90^\circ$ counterclockwise from $e$ and connects the faces which $e$ separates. Notice that each edge $e^*$ corresponds to exactly one edge, so taking the cost on the primal edge $c(\lambda, e)$ is well defined for each dual edge $e^*$. 

\subsection{Erikson's Algorithm}

The algorithm that Erikson provides in \cite{erikson2010} first creates a shortest-paths tree, then runs a parameterized shortest-paths algorithm. It uses a dynamic tree data structure to find pivots and remove them from the tree. Erikson uses a few lemmas to prove the correctness of his algorithm. We list the important ones below (proofs are given in the original paper).

\begin{lemma}
There exists a feasible $s-t$ flow in $G$ with value $\lambda$ if and only if the dual residual network $G^*_\lambda$ does not contain a negative cycle. 
\label{lem:resid-network}
\end{lemma}

\begin{lemma}
$\lambda_{max}$ is the first critical value of $\lambda$ whose pivot introduces a directed cycle into $T_{\lambda}$. 
\label{lem:directed-cycle}
\end{lemma}

We define $\lambda_{max}$ as the maximum value of $\lambda$ for which $G^*_\lambda$ has no negative weight cycles and is well defined. Note that lemma \ref{lem:resid-network} implies that $\lambda_{max}$ is exactly the maximum flow of the original graph $G$ with capacities $c : E \to \mathrm{R}$. To show this, suppose there existed some flow of value $f > \lambda_{max}$. Then, it must be feasible so that by lemma \ref{lem:resid-network}, $G^*_f$ does not contain a negative cycle. However, this implies that $f > \lambda_{max}$ is a value of $\lambda$ for which $G^*_\lambda$ has no negative weight cycles, which is a contradiction.

The algorithm computes $T_\lambda$, which is the shortest paths tree with parameter $\lambda$, computed on $G^*_\lambda$. Notice that using a version of the parametric shortest paths algorithm presented in the above section, one can obtain $\lambda_{max}$ and hence the max-flow of $G$. This can be found using $T_\lambda$ and by increasing $\lambda$ and pivoting edges into $T_\lambda$ until a cycle is found. Using lemma \ref{lem:directed-cycle}, once a directed cycle occurs in $T_{\lambda}$, it must be true that $\lambda$ will be equal to the max flow in $G$.

The overall structure of the algorithm is now set. First, one computes the minimum-distance spanning tree in $G^*_\lambda$. Next, iteratively decrease $\lambda$ and maintain the minimum-distance spanning tree with parameter $\lambda$. Eventually, a directed cycle will be introduced, and when that occurs, $\lambda$ will be equal to the maximum flow value. 

\subsection{Formalized Algorithm}

To make the above overview more formal, the pseudocode for Erikson's algorithm is provided below. First, however, we define the distance $dist(\lambda, p)$ for a vertex $p$ in the dual with parameter $\lambda$ as the shortest path distance from some arbitrary starting node $o$ to $p$. Note that the shortest path distance is computed over the cost function $c(\lambda, e^*) = c(\lambda, e)$ as defined earlier in the paper (each dual edge retains the cost function of its primal). We define slackness on an edge $e^* = u \to v$ in the dual with parameter $\lambda$ as $slack(\lambda, e^*) = dist(\lambda, u) + c(\lambda, e^*) - dist(\lambda, v)$. 

The slackness of an edge in the dual provides a basis for how pivots in the shortest paths tree are introduced. In particular, Erikson notes that the algorithm has an invariant which keeps $slack(\lambda, e^*) = 0$ exactly when $e^* \in T_{\lambda}$ or when $\lambda$ is a critical value corresponding to a new pivot. 

We are now ready to present Erikson's algorithm in pseudocode (taken from Erikson \cite{erikson2010}):
\begin{quote}
\begin{verbatim}
PlanarMaxFlow(G, c, s, t):
    Initialize spanning tree L
    While s and t are in the same component of L:
        LP = path in L from s to t
        min_edge = edge in LP* with minimum slack
        d = slack(min_edge)
        forall edges e in LP:
            slack(e*) = slack(e*) - d
            slack(rev(e*)) = slack(rev(e*)) + d
        L.remove(e*)
        if pred(head(e)) != null:
            L.insert(Edge(head(e),e)*)
        pred(head(e)) = tail(e)
    forall edges e:
        flow(e) = c(e) - slack(e*)
    return flow
\end{verbatim}
\end{quote}

Note that the algorithm runs in $O(n \log n)$ time when one uses dynamic tree data structures for maintaining the spanning tree. For instance, Tarjan and Werneck \cite{tarjanwerneck2005} present top trees which are able to do all necessary spanning tree operations in $O(\log n)$ time. These operations include determining whether two vertices exist in the same component of the spanning tree, finding the minimum weight edge in a path, and finding a path between two vertices (in addition to being able to insert and delete edges from the tree). 

Erikson proves a $O(n \log n)$ runtime for the algorithm by bounding the number of pivots by $O(n)$ (see lemma 2.6 of his paper for more details of the proof). Once the number of pivots is bounded, there are only a constant number of $O(\log n)$ operations per pivot, which implies $O(n \log n)$ total time for the while loop. Initializing the spanning tree requires $O(n)$ for computing the dual and $O(n \log n)$ for computing shortest paths in a planar graph (by using Dijkstra's). The top tree can also be initialized in $O(n \log n)$ time using Tarjan and Werneck's initialization scheme. 

\section{Framework and Infrastructure}

The Erikson algorithm was implemented in Java 1.7 and is available on \url{https://github.com/jrshoch/msmsmaxflow}. The architecture was created so that each piece of code was as modular and extensible as possible. Special emphasis was put on readable and self-documenting code so that the algorithm could be tweaked easily. Since design decisions usually favored writing good code, some computational speed was lost. However, the most noticeable loss of speed due to the object oriented architecture would prevail across all graph algorithms under our particular framework, which means that each algorithm could be compared in relation to each other.

\subsection{Design Decisions}

We chose to implement the Erikson algorithm in Java because of its object-oriented nature. Java represented a middle ground between performance and modularity. Implementation in a lower level language like C or C++ would have made it difficult to generalize the architecture, while a higher level language like Python or Ruby would have been too slow for automated testing at a large scale. Java provides enough object oriented functionality to make our architecture easily extensible and modular, as well as enough speed to be able to run the Erikson and other max-flow algorithms on large graphs.

The graph $G$ and its dual $G^*$ were abstracted away as objects so that all max-flow algorithms that we implemented would be able to operate under the same API. The graph class operates in relation to other objects, each representing a physical portion of a graph. Vertex was a fundamental object, with edges being composed of two vertices. Faces were then constructed by a cycle of vertices and edges. 

Most of the work in the graph was handled by the Graph object. The graph contained information about neighboring vertices and served as a medium for finding and manipulating flows along edges. The original Graph interface created an immutable object so that vertices and edges were predefined and linked up together in the graph. The immutability enabled the algorithm to assume that vertices and edges would never get destroyed, which greatly simplified the amount of work for most algorithms. Moreover, this immutability improved the speed of the underlying data structure since only calls to the structure would be made. Flows were overlayed on top of the graph by creating subclasses which contained information about flow and capacity.  

We tried to demarcate objects and methods in a manner which had natural analogies to graphs. Since a vertex, face, or edge without a graph does not have very much useful information, most of the knowledge was abstracted to the graph level. Neighboring vertices could be found only at the graph level, with each vertex completely unaware of its connections with other vertices. An edge had information about its head and tail vertex, but could not performa any operations.

Flow, slackness, and other additions to the graph and its subcomponents were represented by creating subclasses from the basic interfaces. This enhanced the modularity of the design. For instance, any breadth first search on the graph would only need to know about the vertices and edges, along with their neighbors, and would not be given informatoin about flow or slackness. 

\subsection{Automated Testing}



\newpage

\begin{thebibliography}{9}

    \bibitem{erikson2010}
        Jeff Erikson.
       ``Maximum Flows and Parametric Shortest Paths in Planar Graphs.''
       \emph{Proceedings of the 21st Annual ACM-SIAM Symposium on Discrete Algorithms}.
       794-804.
       2010.

    \bibitem{dinitz1970}
        Yefim Dinitz.
        ``Algorithm for Solution of a Problem of Maximum Flow in a Network with Power Estimation.''
        \emph{Doklady Akademii nauk SSSR}.
        11: 1277-1280.
        1970.

    \bibitem{goldbergtarjan1986}
        Andrew Goldberg and Robert Tarjan.
        ``A New Approach to the Maximum Flow Problem.''
        \emph{Annual ACM Symposium on Theory of Computing}.
        Proceedings of the Eighteenth Annual ACM Symposium on Theory of Computing.
        136-146.
        1986.

    \bibitem{greigporteousseheult1989}
        D.M. Greig, B.T. Porteous, and A.H. Seheult.
        ``Exact Maximum A Posteriori Estimation for Binary Images.''
        \emph{Journal of the Royal Statistical Society. Series B (Methodological).}
        51(2): 271-279.
        1989.

    \bibitem{tarjanwerneck2005}
        R.E. Tarjan and R.F. Werneck.
        ``Self-Adjusting Top Trees.''
        \emph{Annual ACM Symposium on Discrete Algorithms.}
        Proceedings of the Sixteenth Annual ACM Symposium on Discrete Algorithms.
        812-822,
        2005.

    \bibitem{youngtarjanorlin1991}
        N.E. Young, R.E. Tarjan, and J.B. Orlin. 
        ``Faster Parametric Shortest Path and Minimum Balance Algorithms.''
        \emph{Networks.}
        21(2): 205-221.
        1991.

\end{thebibliography}

\end{document}
